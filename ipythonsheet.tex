% Cheat sheet for IPython. Based on latexsheet by Winston Chang [0] and
% Doug Tillmans notes [1] from Jeff Rush's ShowMeDo tutorials [2].
%
% [0] <http://www.stdout.org/~winston/latex/>
% [1] <http://www.dugthawts.com/?p=284>
% [2] <http://showmedo.com/videotutorials/video?name=1000010&fromSeriesID=100>
%
% Copyright: 2009, 2011 Morten Kjeldgaard <mok@bioxray.dk>
% License: CC-BY-SA 3.0, see: <http://creativecommons.org/licenses/by-sa/3.0/>


% Magicks TODO:
%
%Exit 
%Pprint 
%Quit 
%autocall 
%autoindent 
%automagic 
%clear 
%color_info 
%colors 
%debug
%doctest_mode 
%ed 
%env 
%exit 
%paste 
%pdb 
%profile 
%prun
% %quickref 
%quit 
%r 
%rehash 
%rehashx
%rep 
%save 
%system_verbose 
%time
%upgrade 
%who_ls 
%xmode

% -----------------------------------------------------------------------

\documentclass[10pt,landscape]{article}
\usepackage{multicol}
\usepackage{calc}
\usepackage{ifthen}
\usepackage[landscape]{geometry}

% To make this come out properly in landscape mode, do one of the following
% 1.
%  pdflatex ipythonsheet.tex
%
% 2.
%  latex ipythonsheet.tex
%  dvips -P pdf  -t landscape ipythonsheet.dvi
%  ps2pdf ipythonsheet.ps

% This sets page margins to .5 inch if using letter paper, and to 1cm
% if using A4 paper. (This probably isn't strictly necessary.)
% If using another size paper, use default 1cm margins.

\ifthenelse{\lengthtest { \paperwidth = 11in}}
	{ \geometry{top=.5in,left=.5in,right=.5in,bottom=.5in} }
	{\ifthenelse{ \lengthtest{ \paperwidth = 297mm}}
		{\geometry{top=1cm,left=1cm,right=1cm,bottom=1cm} }
		{\geometry{top=1cm,left=1cm,right=1cm,bottom=1cm} }
	}

% Turn off header and footer
\pagestyle{empty}
 

% Redefine section commands to use less space
\makeatletter
\renewcommand{\section}{\@startsection{section}{1}{0mm}%
                                {-1ex plus -.5ex minus -.2ex}%
                                {0.5ex plus .2ex}%x
                                {\normalfont\large\bfseries}}
\renewcommand{\subsection}{\@startsection{subsection}{2}{0mm}%
                                {-1explus -.5ex minus -.2ex}%
                                {0.5ex plus .2ex}%
                                {\normalfont\normalsize\bfseries}}
\renewcommand{\subsubsection}{\@startsection{subsubsection}{3}{0mm}%
                                {-1ex plus -.5ex minus -.2ex}%
                                {1ex plus .2ex}%
                                {\normalfont\small\bfseries}}
\makeatother

% Don't print section numbers
\setcounter{secnumdepth}{0}


\setlength{\parindent}{0pt}
\setlength{\parskip}{0pt plus 0.5ex}


% -----------------------------------------------------------------------

\begin{document}

\raggedright
\footnotesize
\begin{multicols}{3}


% multicol parameters
% These lengths are set only within the two main columns
%\setlength{\columnseprule}{0.25pt}
\setlength{\premulticols}{1pt}
\setlength{\postmulticols}{1pt}
\setlength{\multicolsep}{1pt}
\setlength{\columnsep}{2pt}

\begin{center}
     \Large{\textbf{IPython\ Cheat\ Sheet}} \\
\end{center}

%%%%%%%%%%%%%%%%%%%%%%%%%%%%%%%%%

\section {Magic Commands}

If automagick is enabled, the \% sign in front of magick commands is
optional.

\subsection{Python variables}

\newlength{\MyLen}

\settowidth{\MyLen}{1234567890123456}
\begin{tabular}{@{}p{\MyLen}%
               @{}p{\linewidth-\MyLen}@{}}
  \verb!%who!    & Bring up a list of known variables. \\
  \verb!%who int!  & Show int variables in namespace. \\
  \verb!%whos! & Show all variables, their types and values. \\
  \verb!%psearch! \textit{var*}  & List variables matching ``var*''. \\
  \verb!%psearch! \textit{var* int}  & List variables matching ``var*'' of type int. \\
  \verb!%store! \textit{var} & Store variable. \\
  \verb!%store! \textit{var $>$ file}  & Save variable to a file. \\
  \verb!%store!  & Show all stored variables. \\
  \verb!%store -r!  & Recover stored variables to memory. \\
  \verb!%store -z!  & Zap stored variables.\\
  \verb!%reset!  & Delete all variables in the IPython session. \\
\end{tabular}

Variables stored are persistent and can be recovered in the next IPython session.

%%%%%%%%%%%%%%%%%%%%%%%%%%%%%%%%%

\subsection{Logging}
\settowidth{\MyLen}{123456789012}
\begin{tabular}{@{}p{\the\MyLen}%
               @{}p{\linewidth-\the\MyLen}@{}}
  \verb!%logstart!  & Start logging of session. \\
  \verb!%logoff!  & Pause logging. \\
  \verb!%logon!  & Restart logging after pause. \\
  \verb!%logstop!  & Stop logging and close logfile.\\
  \verb!%logstate!  & Report status of logging.\\
  \verb!%runlog!  & Re-execute log file.\\
\end{tabular}

%%%%%%%%%%%%%%%%%%%%%%%%%%%%%%%%%

\subsection{Aliases}
\begin{tabular}{@{}ll@{}}
  \verb!%alias!  & List defined aliases. \\
  \verb!%unalias!  & Remove alias. \\
%%%  \verb!%alias! \textit{pr echo You said: \%s}  & pr Hello.
\end{tabular}

Built-in aliases: cat, clear, less, mkdir, rmdir, cp, ls, mv, rm.

%%%%%%%%%%%%%%%%%%%%%%%%%%%%%%%%%

\subsection{About magic commands}


\begin{tabular}{@{}p{\the\MyLen}%
                @{}p{\linewidth-\the\MyLen}@{}}
  \verb!%lsmagic!  & List all magic commands. \\             
  \verb!%magic!  & Scrolling help of magic commands. \\ 
  \verb!%lsmagic?!  & Summary information about object. \\             
  \verb!%lsmagic??!  & Detailed information about object. \\             
  \verb!var?!  & Show Python docs of this variable. \\             
\end{tabular}




%%%%%%%%%%%%%%%%%%%%%%%%%%%%%%%%%

\subsection{Shorthands}
\begin{tabular}{@{}p{\the\MyLen}%
                @{}p{\linewidth-\the\MyLen}@{}}
  \verb!%p!  & Short for print. \\             
  \verb!%page!  & Print the object through pretty-printer. \\             
  \verb!f 4, 5!  & Call function \texttt{f(4,5)}. \\             
  \verb!,f 4, 5!  & Call function \texttt{f(``4'', ``5'')}. \\             
  \verb!/f!  & Callable: call function \texttt{f()}. \\             
\end{tabular}

%%%%%%%%%%%%%%%%%%%%%%%%%%%%%%%%%

\subsection{Help subsystem}

IPython integrates with the pydocs system. Parentheses can be omitted.

\settowidth{\MyLen}{1234567890123456}
\begin{tabular}{@{}p{\the\MyLen}%
                @{}p{\linewidth-\the\MyLen}@{}}
   \verb!help(sys)! & Display help on the sys module. \\
   \verb!help('for')! & Display help on keyword. \\
   \verb!help('LISTS')! & Caps shows you the topic. \\
   \verb!help()! & Interactive help subsystem.\\
\end{tabular}

%%%%%%%%%%%%%%%%%%%%%%%%%%%%%%%%%

\subsection{Introspection}

\settowidth{\MyLen}{123456789012345678}
\begin{tabular}{@{}p{\the\MyLen}%
                @{}p{\linewidth-\the\MyLen}@{}}
  \verb!%pdef re.match! &  Inspect definition of function.\\
  \verb!%pdoc re.match! & Display \verb!__doc__! string.\\
  \verb!%pinfo!  & Display pdef and pdoc. \\
  \verb!%psource re.match! & Display source code. \\
  \verb!re.match??!  & Same as pinfo.\\
  \verb!re.match?! & pdef and more.\\
  \verb!%pfile re.match! &  Display source file containing function.\\
  \verb!%edit -x os.path! & Opens source file in preferred editor.\\
\end{tabular}


%%%%%%%%%%%%%%%%%%%%%%%%%%%%%%%%%

\subsection{Directory Navigation}

\settowidth{\MyLen}{12345678901234567890}

\begin{tabular}{@{}p{\the\MyLen}%
                @{}p{\linewidth-\the\MyLen}@{}}
  \verb!%pwd! &  Print working directory.\\
  \verb!%pushd! \textit{path} & Push \textit{path} on stack and cd to it. \\
  \verb!%dirs! & List directories on the stack. \\
  \verb!%cd dirs[1]! & Switch to second directory on the stack. \\
  \verb!%popd! & Pop top directory off stack and cd to it.\\
  \verb!%bookmark! \textit{name} & Bookmark CWD \\
  \verb!%bookmark! \textit{name path}  & Bookmark path. \\
  \verb!%bookmark -l!  & List bookmarks. \\
  \verb!%dhist! & Display the history of directories visited. \\
  \verb!%cd! \textit{bookmark}  & cd to the bookmark. \\
  \verb!%cd -!\textit{number} & cd to the numbered directory. \\
\end{tabular}

%%%%%%%%%%%%%%%%%%%%%%%%%%%%%%%%%

\subsection{Shell commands}

\settowidth{\MyLen}{1234567890123456789}
\begin{tabular}{@{}p{\the\MyLen}%
                @{}p{\linewidth-\the\MyLen}@{}}

  \verb/!ls/ & Run the system \texttt{ls} command.\\
  \verb/!ls -l/ & Run the \texttt{ls -l} command.\\
  \verb/patt = "*.py"/ & Store a pattern in a Python variable.\\
  \verb/!ls -l $patt/ & Use Python variable in the shell.\\
  \verb/!ls -l ${patt+'c'}/ & Use Python expressions in the shell.\\
  \verb/x = !ls/ & Store output of shell command in a list.\\
  \verb/x.s/ & Show list as a string.\\
  \verb/x.n/ & Show list as string with newlines.\\
  \verb/x = !ls *.py | sort/ & Use pipes in the shell.\\
  \verb/%sx / & Run a shell command and capture its output in a dict.\\
\end{tabular}

Example: \verb/a = %sx find . -name "*.py"/

%%%%%%%%%%%%%%%%%%%%%%%%%%%%%%%%%

\section{Running Python programs}

Running a program leaves the variables in the local namespace.

\settowidth{\MyLen}{1234567890123456}
\begin{tabular}{@{}p{\the\MyLen}%
                @{}p{\linewidth-\the\MyLen}@{}}
  \verb/%run/ \textit{prog} & Start script with main method.\\
  \verb/%run -p/ \textit{prog} & Profile program.\\
  \verb/%run -d/ \textit{prog} & starts up the calc program with the debugger.\\
  \verb/%run?/ & List options of \texttt{run}.\\
  \verb/%pycat/ \textit{file} & Pretty print \texttt{file.py}.\\
  \verb/%timeit/ \textit{expr}&  Time execution of Python expression.\\
  \verb/%bg/ & Run a job in the background \\
\end{tabular}

%%%%%%%%%%%%%%%%%%%%%%%%%%%%%%%%%

\subsection{Viewing and editing source}

\begin{tabular}{@{}p{\the\MyLen}%
                @{}p{\linewidth-\the\MyLen}@{}}

  \verb/%edit def.py/ & Edit and execute file def.py.\\
  \verb/%edit 1-4 6 / & Edit and execute lines 1--4 and 6.\\
  \verb/%save apple 1-3/ & Save lines 1--3 of history to file apple.py.\\
  \verb/%psource xyz/ & Print source of object xyz.\\
  \verb/%cpaste/ & Paste in source from outside source.\\
\end{tabular}

%%%%%%%%%%%%%%%%%%%%%%%%%%%%%%%%%

\section{History caching}

\begin{tabular}{@{}p{\the\MyLen}%
                @{}p{\linewidth-\the\MyLen}@{}}

  \verb/_i/ &  Show last command.\\
  \verb/_ii/ & Shows two commands back.\\
  \verb/_iii/ & Shows three commands back.\\
  \verb/_i3/ & Show command number 3 entered.\\
  \verb/In[3]/ & global variable showing third command.\\
  \verb/%hist/ & Show last 20 commands entered.\\
  \verb/%hist 1 10/ & Show commands 1--10.\\
  \verb/%hist -r/ & Hide magic wrapper commands.\\
  \verb/%hist -r 10/ & Show last 10 commands.\\
\end{tabular}
\settowidth{\MyLen}{12345678901234567890}
\begin{tabular}{@{}p{\the\MyLen}%
                @{}p{\linewidth-\the\MyLen}@{}}
  \verb/%macro/ \textit{name} \textit{range} & Define history macro.\\
  \verb/%macro begin 1-4 5/ & Define history named ``begin''.\\
  \verb/begin/ & Execute macro ``begin''.\\
  \verb/%store begin/ & Store macro ``begin''.\\
\end{tabular}

For example, execute command number 3: \verb/exec _i3/ \\
Execute commands 1 to 3: \verb/exec In[1:4]/.

%%%%%%%%%%%%%%%%%%%%%%%%%%%%%%%%%

\section{Output caching}

\verb/Out/ is a global dictionary for the output of IPython commands.

\settowidth{\MyLen}{123456789012}
\begin{tabular}{@{}p{\the\MyLen}%
                @{}p{\linewidth-\the\MyLen}@{}}

  \verb/Out.keys()/ & Command numbers with assocated output.\\
  \verb/Out[130]/ & Display output of the command 130.\\

\end{tabular}

%%%%%%%%%%%%%%%%%%%%%%%%%%%%%%%%%






% -----------------------------------------------------------------------

\rule{0.3\linewidth}{0.25pt}
\scriptsize

Copyright \copyright\ 2011 Morten Kjeldgaard \verb!<mok@bioxray.dk>!

\end{multicols}
\end{document}
