\documentclass[10pt,landscape]{article}
\usepackage{multicol}
\usepackage{calc}
\usepackage{ifthen}
\usepackage[landscape]{geometry}

% To make this come out properly in landscape mode, do one of the following
% 1.
%  pdflatex ipythonsheet.tex
%
% 2.
%  latex ipythonsheet.tex
%  dvips -P pdf  -t landscape ipythonsheet.dvi
%  ps2pdf ipythonsheet.ps

% 2009-12-22
% Cheat sheet for ipython. Based on latexsheet by

% 2008-04
% Changed page margin code to use the geometry package. Also added code for
% conditional page margins, depending on paper size. Thanks to Uwe Ziegenhagen
% for the suggestions.

% 2006-08
% Made changes based on suggestions from Gene Cooperman. <gene at ccs.neu.edu>


% To Do:
% \listoffigures \listoftables
% \setcounter{secnumdepth}{0}


% This sets page margins to .5 inch if using letter paper, and to 1cm
% if using A4 paper. (This probably isn't strictly necessary.)
% If using another size paper, use default 1cm margins.
\ifthenelse{\lengthtest { \paperwidth = 11in}}
	{ \geometry{top=.5in,left=.5in,right=.5in,bottom=.5in} }
	{\ifthenelse{ \lengthtest{ \paperwidth = 297mm}}
		{\geometry{top=1cm,left=1cm,right=1cm,bottom=1cm} }
		{\geometry{top=1cm,left=1cm,right=1cm,bottom=1cm} }
	}

% Turn off header and footer
\pagestyle{empty}
 

% Redefine section commands to use less space
\makeatletter
\renewcommand{\section}{\@startsection{section}{1}{0mm}%
                                {-1ex plus -.5ex minus -.2ex}%
                                {0.5ex plus .2ex}%x
                                {\normalfont\large\bfseries}}
\renewcommand{\subsection}{\@startsection{subsection}{2}{0mm}%
                                {-1explus -.5ex minus -.2ex}%
                                {0.5ex plus .2ex}%
                                {\normalfont\normalsize\bfseries}}
\renewcommand{\subsubsection}{\@startsection{subsubsection}{3}{0mm}%
                                {-1ex plus -.5ex minus -.2ex}%
                                {1ex plus .2ex}%
                                {\normalfont\small\bfseries}}
\makeatother

% Don't print section numbers
\setcounter{secnumdepth}{0}


\setlength{\parindent}{0pt}
\setlength{\parskip}{0pt plus 0.5ex}


% -----------------------------------------------------------------------

\begin{document}

\raggedright
\footnotesize
\begin{multicols}{3}


% multicol parameters
% These lengths are set only within the two main columns
%\setlength{\columnseprule}{0.25pt}
\setlength{\premulticols}{1pt}
\setlength{\postmulticols}{1pt}
\setlength{\multicolsep}{1pt}
\setlength{\columnsep}{2pt}

\begin{center}
     \Large{\textbf{IPython\ Cheat\ Sheet}} \\
\end{center}

\section {Magic Commands}
\subsection{Python variables}

\newlength{\MyLen}
\settowidth{\MyLen}{xxxxxxxxxxxx}


\begin{tabular}{@{}p{\MyLen}%
               @{}p{\linewidth-\MyLen}@{}}

  \verb!who!    & Bring up a list of known variables. \\
  \verb!who int!  & Show int variables in namespace. \\
  \verb!whos! & Show all variables, their types and values. \\
  \verb!psearch! \textit{var*}  & List variables matching ``var*''. \\
  \verb!psearch! \textit{var* int}  & List variables matching ``var*'' of type int. \\
  \verb!store! \textit{var} & Store variable. \\
  \verb!store! \textit{var $>$ file}  & Save variable to a file. \\
  \verb!store!  & Show all stored variables. \\
  \verb!store -r!  & Recover stored variables to memory. \\
  \verb!store -z!  & Zap stored variables.\\
  \verb!reset!  & Delete all variables in the ipython session.

\end{tabular}
Variables stored are persistent and can be recovered in the next ipython session.


\subsection{Logging}
%\newlength{\MyLen}
\settowidth{\MyLen}{xxxxxxxxxxxx}

\begin{tabular}{@{}p{\the\MyLen}%
               @{}p{\linewidth-\the\MyLen}@{}}
  \verb!logon!  & Start logging. \\
  \verb!logoff!  & Stop logging. \\
  \verb!logstart!  & Start logging of session. \\
  \verb!logstate!  & Report status of logging.\\
  \verb!runlog!  & Reexecute log file.\\

\end{tabular}


\subsection{Aliases}

\begin{tabular}{@{}ll@{}}
  \verb!alias!  & Display defined aliases. \\
  \verb!unalias!  & Remove alias. \\
  \verb!alias! \textit{pr echo You said: \%s}  & pr Hello.
\end{tabular}

\subsection{Magic commands}
\settowidth{\MyLen}{\texttt{multicol} }
\begin{tabular}{@{}p{\the\MyLen}%
                @{}p{\linewidth-\the\MyLen}@{}}
  \verb!lsmagic!  & List all magic commands. \\             
  \verb!magic!  & Scrolling help of magic commands. \\ 
  \verb!lsmagic?!  & Summary information about object. \\             
  \verb!lsmagic??!  & Detailed information about object. \\             
  \verb!var?!  & Show python docs of this variable. \\             

\end{tabular}
Magic commands have a \% sign in front or not. If you have a variable
of the same name as a magic command there is a difference. 

\subsection{Shorthands}
\settowidth{\MyLen}{\texttt{multicol} }
\begin{tabular}{@{}p{\the\MyLen}%
                @{}p{\linewidth-\the\MyLen}@{}}
  \verb!p!  & Short for print. \\             
  \verb!page!  & Print the object through pretty-printer. \\             
  \verb!f 4, 5!  & Call function \texttt{f(4,5)}. \\             
  \verb!,f 4, 5!  & Call function \texttt{f(``4'', ``5'')}. \\             
  \verb!/f!  & Callable: call function \texttt{f()}. \\             


\end{tabular}


\subsection{Help subsystem}

IPython integras with the pydocs system.

\settowidth{\MyLen}{\texttt{multicol} }
\begin{tabular}{@{}p{\the\MyLen}%
                @{}p{\linewidth-\the\MyLen}@{}}

   \verb!help(sys)! & Display help on the sys module. \\
   \verb!help('for')! & Display help on keyword. \\
   \verb!help('FOR')! & Caps shows you the topic. \\
   \verb!help()! & Interactive help subsystem.
              
\end{tabular}

%%%%%%%%%

\subsection{Introspection}

\settowidth{\MyLen}{\texttt{multicol} }
\begin{tabular}{@{}p{\the\MyLen}%
                @{}p{\linewidth-\the\MyLen}@{}}

  \verb!pdef re.match! &  Inspect definition of function.\\
  \verb!pdoc re.match! & Display \verb!__doc__! string.\\
  \verb!pinfo!  & Display pdef and pdoc. \\
  \verb!psource re.match! & Display source code. \\
  \verb!re.match??!  & Same as pinfo.\\
  \verb!re.match?! & pdef and more.\\
  \verb!pfile re.match! &  Display source file containing function.\\
  \verb!edit -x re.match! & Opens source file in preferred editor.\\
              
\end{tabular}



\subsection{Directory Navigation}

\settowidth{\MyLen}{\texttt{multicol} }
\begin{tabular}{@{}p{\the\MyLen}%
                @{}p{\linewidth-\the\MyLen}@{}}

  \verb!pwd! &  Print working directory.\\
  \verb!pushd! \textit{path} &  Push \textit{path} on stack and cd to it. \\
  \verb!dirs! & List directories on the stack. \\
  \verb!cd dirs[1]! & Switch to second directory on the stack. \\
  \verb!popd! & Pop top directory off the the stack and cd to it.\\
  \verb!bookmark! \textit{name} & Bookmark CWD \\
  \verb!bookmark! \textit{name path}  & Bookmark path. \\
  \verb!bookmark -l!  & List bookmarks. \\
  \verb!dhist! & Display the history of directories visited. \\
  \verb!cd! \textit{bookmark}  & cd to the bookmark. \\
  \verb!cd -!\textit{number} & cd to the numbered directory.

\end{tabular}

\subsection{Shell commands}

\settowidth{\MyLen}{\texttt{multicol} }
\begin{tabular}{@{}p{\the\MyLen}%
                @{}p{\linewidth-\the\MyLen}@{}}

  \verb/!ls/ & Run the \texttt{ls} command.\\
  \verb/!ls -l/ & Run the \texttt{ls -l} command.\\
  \verb/patt = "*.py"/ & Store a pattern in a Python variable.\\
  \verb/!ls -l \$patt/ & Use python variable in the shell.\\
  \verb/!ls -l \${patt+'c'}/ & Use python expressions in the shell.\\
  \verb/x = !ls/ & Store output of shell command in a list.\\
  \verb/x.s/ & Show list as a string.\\
  \verb/x.n/ & Show list as string with newlines.\\
  \verb/x = !ls *.py | sort/ & Use pipes in the shell.\\

\end{tabular}

\section{Running Python programs}

Running a program leaves the variables in the local namespace.

\settowidth{\MyLen}{\texttt{multicol} }
\begin{tabular}{@{}p{\the\MyLen}%
                @{}p{\linewidth-\the\MyLen}@{}}

\verb/run/ \textit{prog} & Start script with main method.\\
\verb/run -p/ \textit{prog} & Profile program.\\
\verb/run -d/ \textit{prog} & starts up the calc program with the debugger.\\
\verb/run?/ & List options of \texttt{run}.\\
\verb/pycat file/ & Pretty print \texttt{file.py}.\\

\end{tabular}

\section{iPython history caching}

Ctrl-P, Ctrl-N, tab completion, etc. Arrow keys.

\settowidth{\MyLen}{\texttt{multicol} }
\begin{tabular}{@{}p{\the\MyLen}%
                @{}p{\linewidth-\the\MyLen}@{}}


\verb/\_i/ &  Show last command.\\
\verb/\_ii/ & Shows two commands back.\\
\verb/\_i3/ & Show command number 3 entered.\\
\verb/In[3]/ & global variable that shows third command.\\
\verb/history/ or \verb/hist/ & Show last 20 commands entered.\\
\verb/hist 1 10/ & Show commands 1--10.\\
\verb/hist -r/ & Hide magic wrapper commands.\\
\verb/hist -r 10/ & Show last 10 commands.\\
\verb/exec _i3/ & Execute command number 3.\\
\verb/exec In[1:4]/ & Execute commands 1 to 3.\\
\verb/macro/ \textit{name} \textit{range} & Define history macro.\\
\verb/macro begin 1-4 5/ & Define history named ``begin''.\\
\verb/begin/ & Execute macro ``begin''.\\
\verb/store begin/ & Store macro ``begin''.\\

\end{tabular}


\section{Viewing and editing source}


\settowidth{\MyLen}{\texttt{multicol} }
\begin{tabular}{@{}p{\the\MyLen}%
                @{}p{\linewidth-\the\MyLen}@{}}

\verb/edit def.py/ & Edit and execute file def.py.\\
\verb/edit 1-4 6 / & Edit and execute lines 1--4 and 6.\\
\verb/save apple 1-3/ & Save lines 1--3 of history to file apple.py.\\
\verb/psource xyz/ & Print source of object xyz.\\

\verb/cpaste/ & Paste in source from outside source.\\

\end{tabular}

\section{Output caching}

Out is global dictionary for the output of iPython commands.

\settowidth{\MyLen}{\texttt{multicol} }
\begin{tabular}{@{}p{\the\MyLen}%
                @{}p{\linewidth-\the\MyLen}@{}}

\verb/Out.keys()/ & Command numbers with assocated output.\\

\verb/Out[130]/ & Display output of the command 130.\\

\end{tabular}





              








\rule{0.3\linewidth}{0.25pt}
\scriptsize

Copyright \copyright\ 2009 Morten Kjeldgaard.

\end{multicols}
\end{document}
